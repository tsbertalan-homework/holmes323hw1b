% \documentclass[letterpaper,10pt]{article}
\documentclass{tufte-handout}
\setcounter{secnumdepth}{2}
\usepackage[utf8]{inputenc}
% \usepackage[margin=1.5cm]{geometry}32

\usepackage{graphicx,homework,minted}


\title{MAT351 - HW1 - Part 2}
\author{Tom Bertalan}


\begin{document}

\maketitle
\tableofcontents

\newpage
\section{Problem 5: Wilson Exercise 5.1, p70, et al.}
\begin{figure}[ht]
 \includegraphics[width=\columnwidth]{code/hw1b-5-adaptive.pdf}
 \caption{Adaptive timestepping vs widely-stepped RK4. Since Matlab's 
\texttt{ODE45} and \texttt{ODE15s} algorithms, and SciPy's equivalent
\texttt{dopri5} and \texttt{vode/bdf} actually don't do 
vanilla RK4, but a more sophisticated adaptive method, I thought it would be 
interesting to show this graphically. The same timestep is given to both 
solvers, but the ode15s-analog adaptively adds extra timesteps where the 
rate-of-change is high.\label{fig:adaptive}}
\end{figure}
\begin{figure}[ht]
 \includegraphics[width=\columnwidth]{code/hw1b-wils5_1.pdf}
 \caption{Wilson, problem 5.1. simple Naka-Rushton neuron via Runge-Kutta 
$\mathcal O(4)$ time-integration.
Only the last 0.20 [s] of integration time is shown.
Legend gives timestep in [ms].
Analytical solution (not shown) for $P(t)=1$ (constant forcing) is $R(t) = 
\frac{50}{13}(1 - e^{-50t})$
\label{fig:wils}}
\end{figure}
\begin{figure}[ht]
 \includegraphics[width=\columnwidth]{code/hw1b-5-forwardBackward.pdf}
 \caption{Naka-Rushton neuron via forward and backward Euler-integration.
Only the last 0.20 [s] of integration time is shown.
Legend gives timestep in [ms]. All but the largest of the timesteps seem to be 
in the domain of stability for backwards Euler for this problem.
\label{fig:fb}}
\end{figure}
\begin{figure}[ht]
 \includegraphics[width=\columnwidth]{code/hw1b-5-error.pdf}
 \caption{Growth of RK4 error as stepsize $h$ increases. Error is measured as 
the $L_2$ norm of the difference between the solution for $h=10^{-5}$ [s] 
(assumed to be the ``true'' solution) and a linear interpolant of the 
solution $R(t)$ for the given value of $h$. In contrast to the theoretical 
prediction of $r\sim\mathcal O(4)$, I regressed an exponent better than 2 for 
this problem (below the limiting $h$ value where RK4 diverged). This is 
consistent, I think with the $\mathcal O(4)$ estimate being an upper bound on 
the error.
\label{fig:fb}}
\end{figure}
\newpage

\inputminted[]{python}{code/hw1bWilsonC5p1.py}

\newpage
\section{Problem 6: Numerical experiments on a nonliner ODE}
\begin{figure}[ht]
 \includegraphics[width=\columnwidth]{code/hw1bp6-flows.pdf}
 \caption{Partial phase portraits for a Fitzhugh-Nagumo neuron for several 
values of the applied-current parameter $I_\mathrm{app}$. Green points are 
initial conditions; red are final (after 10 [s] of integration). As also 
happens in the Hodgkin-Huxley neuron, there is a lower and an upper steady 
state, separated from a stable limit cycle by a pair of Hopf bifurcations. For 
$I_\mathrm{app}=0$, the fixed point is at $v=\frac{-3}{2}$, $r=\frac{-3}{8}$.
\label{fig:fb}}
\end{figure}

\inputminted[]{python}{code/hw1bp6.py}

\newpage
\section{Integrators.py}
\inputminted[]{python}{code/Integrators.py}



% \begin{marginfigure}
% \centering
% \begin{python}
% import kevrekidis as kv
% import numpy as np
% f, a = kv.fa()
% X = np.random.normal(size=(100,))
% Y = np.random.normal(size=(100,))
% a.scatter(X, Y)
% fname = 'hw1b-scatter.pdf'
% f.savefig(fname)
% print '\includegraphics[width=1.0\\columnwidth]{%s}' % fname
% # kv.plotting.show()
% \end{python}
% \caption{100 points binormal
% \label{fig:1}}
% \end{marginfigure}


\end{document}
